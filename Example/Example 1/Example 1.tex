\documentclass{article}
%
\usepackage{luacode}
\usepackage[fleqn]{amsmath}
\usepackage{amssymb,amstext}
%
\begin{document}
\begin{luacode}
	--load tVar library
	require("../../tVar.lua")
	
	--global Definitions
	tVar.outputMode = "RES_EQ_N"
	tVar.numFormat = "%.1f"
	tVar.numeration = false
	tVar.debugMode = "on"
	
	
	--start fun
	a_1 = tVar:New(3,"a_1"):outRES()
	a_2 = tVar:New(11.4,"a_2"):outRES()
	gamma_1_2 = tVar:New(1.3,"\\gamma_{1,2}"):outRES()
	
	res = ((a_1+a_2)/gamma_1_2+a_1^2):setName("res"):print()
	
	res2 = (res:clean() - a_1^2):setName(""):setUnit("kg"):print()
\end{luacode}
\end{document}